\documentclass{article}
\usepackage[english, russian]{babel}
\babelfont{rm}{cmunrm.otf}

\usepackage{amsmath}
\usepackage{relsize}
\usepackage{mathtools}

\usepackage{geometry}
\geometry{a4paper, left=1.2in, right=1in}

\usepackage{titlesec}
\titleformat{\section}
  {\normalfont\fontsize{21}{18}\bfseries}{\thesection}{1em}{}

\usepackage[bookmarks=false]{hyperref}
\hypersetup{pdfstartview={FitH},
            colorlinks=true,
            linkcolor=magenta,
            pdfauthor={Павел Соболев}}

\usepackage{caption}
\captionsetup[figure]{labelfont=bf,textfont=normalfont,singlelinecheck=off,justification=raggedright}
\setlength{\captionmargin}{1pt}

\usepackage{float}
\input{../../shared/julia_font}
\input{../../shared/julia_listings}
\lstset{language=Julia, style=julia}

\setlength\parindent{0pt}

\newcommand{\hl}[1]{(\hyperlink{eq:#1}{#1})}

\newcommand{\s}[1]{\iter\hypertarget{skip:\thesecounter.\theeqcounter}{\vspace{#1pt}}}
\newcommand{\sd}[0]{\iter\hypertarget{skip:\thesecounter.\theeqcounter}{\vspace{-10pt}}}

\newcommand{\hep}[1]{\vspace{#1pt}\hypertarget{eq:\thesecounter.\theeqcounter}{}\vspace{-#1pt}}

\newcommand{\hs}[0]{\sd\hep{18}}
\newcommand{\hst}[0]{\sd\hep{22}}
\newcommand{\hsv}[1]{\vspace{-#1pt}\sd\hep{#1}}

\numberwithin{equation}{section}
\newcounter{secounter}[section]
\newcounter{eqcounter}
\newcommand{\iter}[0]{\stepcounter{eqcounter}}
\newcommand{\resetcounters}[1]{\setcounter{section}{#1}
\setcounter{subsection}{0}
\setcounter{equation}{0}
\setcounter{secounter}{#1}
\setcounter{eqcounter}{0}}

\hypersetup{pdftitle={Вычислительный практикум: отчет по третьему заданию}}

\usepackage{graphicx}

\begin{document}

\pagenumbering{Alph}
\thispdfpagelabel{Title}
\begin{titlepage}

    \newcommand{\HRule}{\rule{\linewidth}{0.5mm}}

    \center

    \ \\[6.5cm]

    \textsc{\Large Вычислительный практикум}\\[0.5cm]

    \HRule\\[0.4cm]

    {\huge\bfseries Отчет по третьему заданию}\\[0.4cm]

    \HRule\\[0.5cm]

    \large
    \textit{Автор:} Павел \textsc{Соболев}

    \ \\[0.9cm]
    \vfill\vfill\vfill

    {\large\today}

    \vfill

\end{titlepage}
\pagenumbering{arabic}

\subsection*{Часть 1}
\section*{Теория}
\resetcounters{1}

\vspace{18pt}

Дана одна или несколько функций $ a_{ij} $, задающих элементы симметричной матрицы. Для этой матрицы необходимо найти

\begin{itemize}
    \setlength{\itemsep}{1pt}
    \item наибольшее по модулю собственное число $ \lambda_{max} $ степенным методом;
    \item наименьшее по модулю собственное число $ \lambda_{min} $ обратным степенным методом;
    \item все собственные числа, используя классический метод Якоби.
\end{itemize}

\subsection{Степенной метод}

Выбрав начальный вектор $ x_0 $, проводим итерации следующего вида

\hs
\begin{gather}
     \tilde{x}_{n+1} = A x_n, \; x_{n+1} = \frac{\tilde{x}_{n+1}}{{\{\tilde{x}_{n+1}\}}_1}.
\end{gather}

В процессе $ {\{\tilde{x}_{n+1}\}}_1 \longrightarrow \lambda_{max} $.

\subsection{Обратный степенной метод}

Выбрав начальный вектор $ x_0 $, проводим итерации следующего вида

\hs
\begin{gather}
     A \tilde{x}_{n+1} = x_n, \; x_{n+1} = \frac{\tilde{x}_{n+1}}{{\{\tilde{x}_{n+1}\}}_1}.
\end{gather}

Решая системы линейных уравнений, в процессе $ {\{\tilde{x}_{n+1}\}}_1 \longrightarrow \lambda_{min} $.

\subsection{Классический метод Якоби}

Пусть $ A $ -- симметричная матрица, а $ G = G(i, j, \theta) $ -- матрица вращения. Тогда матрица $ A' = G^\top A G $ симметрична и подобна матрице A. Вращение выражается как

\hs
\begin{gather}
A'_{ii} = c^2 A_{ii} - 2 sc \, A_{ij} + s^2 A_{jj},
\end{gather}

\hsv{14}
\begin{gather}
A'_{jj} = s^2 A_{ii} + 2 sc \, A_{ij} + c^2 A_{jj},
\end{gather}

\hsv{14}
\begin{gather}
A'_{ij} = A'_{ji} = (c^2 - s^2) A_{ij} + sc \, (A_{ii} - A_{jj}),
\end{gather}

\hsv{14}
\begin{gather}
A'_{ik} = A'_{ki} = c \, A_{ik} - s \, A_{jk}, \; k \neq i, j,
\end{gather}

\hsv{14}
\begin{gather}
A'_{jk} = A'_{kj} = s \, A_{ik} + c \, A_{jk}, \; k \neq i, j,
\end{gather}

\hsv{14}
\begin{gather}
A'_{kl} = A_{kl}, \; k, l \neq i, j,
\end{gather}

где $ s = \sin \theta $ и $ c = \cos \theta $, причем можем выбрать $ \theta $ так, что $ A'_{ij} = 0 $:

\hs
\begin{gather}
\tg(2 \theta) = \frac{2 A_{ij}}{A_{jj} - A_{ii}}.
\end{gather}

Если $ A_{jj} = A_{ii} $, то

\hs
\begin{gather}
    \theta = \frac{\pi}{4}.
\end{gather}

Для оптимального эффекта $ A_{ij} $ выбирается как наибольший по модулю внедиагональный элемент. Вращения продолжаются до тех пор, пока матрица не приобретет диагональный вид.

\newpage

\subsection*{Часть 2}
\section*{Реализация}
\resetcounters{2}

\vspace{18pt}

Алгоритм реализован на языке программирования \href{https://julialang.org}{\footnotesize \texttt{Julia}} в виде локального модуля {\footnotesize \texttt{A3}} и расположен в GitHub репозитории \href{https://github.com/paveloom-p/P12}{\footnotesize \texttt{paveloom-p/P12}} в папке \href{https://github.com/paveloom-p/P12/tree/master/A3}{\footnotesize \texttt{A3}}. Для воспроизведения результатов следуй инструкциям в файле {\footnotesize \texttt{README.md}}. Далее приводятся только сниппеты кода.

\subsection{Степенной метод}

\begin{figure}[H]
\begin{lstlisting}[
    caption=Поиск наибольшего по модулю собственного числа
]
# Get the matrix
A = Symmetric(Array{Float64}(undef, n, n))
for i = 1:n, j = i:n
    A.data[i,j] = aᵢⱼ(i, j)
end

# Get a vector of initial values
x₀ = get_x₀(A)

# Find the maximum modulo eigenvalue

xₙ = copy(x₀)
λₘₐₓ = 0
λₚᵣₑᵥ = 1

while abs(1 - λₘₐₓ / λₚᵣₑᵥ) > ϵₚ
    λₚᵣₑᵥ = λₘₐₓ
    x̃ₙ₊₁ = A * xₙ
    λₘₐₓ = x̃ₙ₊₁[1]
    xₙ = x̃ₙ₊₁ / λₘₐₓ
end
\end{lstlisting}
\end{figure}

Конструктор типа {\footnotesize \texttt{Symmetric}} позволяет получить вид на верхний или нижний треугольник симметричной матрицы и предоставляет доступ к соответствующим оптимизациям. Интерфейс, однако, ожидает передачу всей матрицы, хотя и необязательно полностью инициализированной. \par

\vspace{\baselineskip}

Функция {\footnotesize \texttt{get\_x₀}} реализует получение вектора-столбца начальных значений. По умолчанию это вектор из единиц.

\subsection{Обратный степенной метод}

\begin{figure}[H]
\begin{lstlisting}[
    caption=Поиск наименьшего по модулю собственного числа
]
# Find the minimum modulo eigenvalue

xₙ = copy(x₀)
λ⁻¹ₘᵢₙ = 0
λ⁻¹ₚᵣₑᵥ = 1

while abs(1 - λ⁻¹ₘᵢₙ / λ⁻¹ₚᵣₑᵥ) > ϵₚ
    λ⁻¹ₚᵣₑᵥ = λ⁻¹ₘᵢₙ
    x̃ₙ₊₁ = A \ xₙ
    λ⁻¹ₘᵢₙ = x̃ₙ₊₁[1]
    xₙ = x̃ₙ₊₁ / λ⁻¹ₘᵢₙ
end
\end{lstlisting}
\end{figure}

\subsection{Классический метод Якоби}

\begin{figure}[H]
\begin{lstlisting}[
    caption=Поиск всех собственных чисел (часть 1)
]
# Use the Jacobi eigenvalue algorithm to find all eigenvalues
λ = Vector{Float64}(undef, n)

for _ in 1:iterₘₐₓ

    aₘₐₓ = k = l = 0

    for i in 1:n-1, j in i+1:n
        if abs(A[i,j]) ≥ aₘₐₓ
            aₘₐₓ = abs(A[i,j])
            k = i; l = j
        end
    end

    if aₘₐₓ < ϵⱼ
        λ = Diagonal(A).diag
        break
    end

    Δ = A[l,l] - A[k,k]

    t = if abs(A[k,l]) < abs(Δ) * 1e-36
        A[k,l] / Δ
    else
        ϕ = Δ / (2 * A[k,l])
        if ϕ < 0
            -1 / (abs(ϕ) + √(ϕ^2 + 1))
        else
            1 / (abs(ϕ) + √(ϕ^2 + 1))
        end
    end

    <...>

\end{lstlisting}
\end{figure}

\begin{figure}[H]
\begin{lstlisting}[
    caption=Поиск всех собственных чисел (часть 2)
]
    <...>

    c = 1 / √(t^2 + 1)
    s = t * c
    τ = s / (1 + c)

    aₜₑₘₚ = A[k,l]
    A.data[k,l] = 0
    A.data[k,k] -= t * aₜₑₘₚ
    A.data[l,l] += t * aₜₑₘₚ

    for i in 1:k-1
        aₜₑₘₚ = A[i,k]
        A.data[i,k] = aₜₑₘₚ - s * (A[i,l] + τ * aₜₑₘₚ)
        A.data[i,l] += s * (aₜₑₘₚ - τ * A[i,l])
    end

    for i in k+1:l-1
        aₜₑₘₚ = A[k,i]
        A.data[k,i] = aₜₑₘₚ - s * (A[i,l] + τ * A[k,i])
        A.data[i,l] += s * (aₜₑₘₚ - τ * A[i,l])
    end

    for i in l+1:n
        aₜₑₘₚ = A[k,i]
        A.data[k,i] = aₜₑₘₚ - s * (A[l,i] + τ * aₜₑₘₚ)
        A.data[l,i] += s * (aₜₑₘₚ - τ * A[l,i])
    end

end
\end{lstlisting}
\end{figure}

Изменение данных типа {\footnotesize \texttt{Symmetric}} в текущей версии требует доступа к полю {\footnotesize \texttt{data}}. Изменение одного элемента, однако, автоматически изменяет и симметричный ему элемент.

\subsection{Пример}

Симметричная матрица задана функцией

\hs
\begin{gather}
    a_{ij} = \frac{(-1)^{i + j}}{\left| i - j \right| + 1}.
\end{gather}

\begin{figure}[H]
\begin{lstlisting}[
    caption=Определение и решение проблемы
]
include("src/A3.jl")
using .A3

λₘₐₓ, λₘᵢₙ, λ = solve(
    Problem(
        function(i, j)
            return (-1)^(i+j) / (abs(i-j) + 1)
        end,
    ),
)
\end{lstlisting}
\end{figure}

\begin{figure}[H]
\begin{lstlisting}[
    caption=Результат
]
λₘₐₓ = 3.478950198080321
λₘᵢₙ = 0.3915295599251142

λ:
1.653468417452173
0.48334088187775437
0.6592071525661884
0.4078747606964994
1.1122936734459035
0.43707220669552194
0.39152955992511285
0.8237636815142848
3.478950198080321
0.5524994677462397
\end{lstlisting}
\end{figure}

\end{document}
