\documentclass{article}
\usepackage[english, russian]{babel}
\babelfont{rm}{cmunrm.otf}

\usepackage{amsmath}
\usepackage{relsize}
\usepackage{mathtools}

\usepackage{geometry}
\geometry{a4paper, left=1.2in, right=1in}

\usepackage{titlesec}
\titleformat{\section}
  {\normalfont\fontsize{21}{18}\bfseries}{\thesection}{1em}{}

\usepackage[bookmarks=false]{hyperref}
\hypersetup{pdfstartview={FitH},
            colorlinks=true,
            linkcolor=magenta,
            pdfauthor={Павел Соболев}}

\usepackage{caption}
\captionsetup[figure]{labelfont=bf,textfont=normalfont,singlelinecheck=off,justification=raggedright}
\setlength{\captionmargin}{1pt}

\usepackage{float}
\input{../../shared/julia_font}
\input{../../shared/julia_listings}
\lstset{language=Julia, style=julia}

\setlength\parindent{0pt}

\newcommand{\hl}[1]{(\hyperlink{eq:#1}{#1})}

\newcommand{\s}[1]{\iter\hypertarget{skip:\thesecounter.\theeqcounter}{\vspace{#1pt}}}
\newcommand{\sd}[0]{\iter\hypertarget{skip:\thesecounter.\theeqcounter}{\vspace{-10pt}}}

\newcommand{\hep}[1]{\vspace{#1pt}\hypertarget{eq:\thesecounter.\theeqcounter}{}\vspace{-#1pt}}

\newcommand{\hs}[0]{\sd\hep{18}}
\newcommand{\hst}[0]{\sd\hep{22}}
\newcommand{\hsv}[1]{\vspace{-#1pt}\sd\hep{#1}}

\numberwithin{equation}{section}
\newcounter{secounter}[section]
\newcounter{eqcounter}
\newcommand{\iter}[0]{\stepcounter{eqcounter}}
\newcommand{\resetcounters}[1]{\setcounter{section}{#1}
\setcounter{subsection}{0}
\setcounter{equation}{0}
\setcounter{secounter}{#1}
\setcounter{eqcounter}{0}}

\hypersetup{pdftitle={Вычислительный практикум: отчет по второму заданию}}

\usepackage{graphicx}

\begin{document}

\pagenumbering{Alph}
\thispdfpagelabel{Title}
\begin{titlepage}

    \newcommand{\HRule}{\rule{\linewidth}{0.5mm}}

    \center

    \ \\[6.5cm]

    \textsc{\Large Вычислительный практикум}\\[0.5cm]

    \HRule\\[0.4cm]

    {\huge\bfseries Отчет по второму заданию}\\[0.4cm]

    \HRule\\[0.5cm]

    \large
    \textit{Автор:} Павел \textsc{Соболев}

    \ \\[0.9cm]
    \vfill\vfill\vfill

    {\large\today}

    \vfill

\end{titlepage}
\pagenumbering{arabic}

\subsection*{Часть 1}
\section*{Теория}
\resetcounters{1}

\vspace{18pt}

Необходимо решить задачу Коши для жесткой системы уравнений

\hs
\begin{gather}
     y'(x) = \alpha(A) y(x) + \beta(A) z(x) + p(x, z(x)),
\end{gather}

\hsv{17.5}
\begin{gather}
     z'(x) = \gamma(A) y(x) + \delta(A) z(x) + q(x, y(x), z(x))
\end{gather}

на промежутке $ [a, b] $ с начальными данными

\hs
\begin{gather}
     y(a) = y_0, \; z(a) = z_0.
\end{gather}

Выберем равномерную сетку $ \{x_k\} $:

\hs
\begin{gather}
x_k = a + kh, \; k \in \{0, \dots, n\}, \; h = (b-a)/n.
\end{gather}

Применим неявный метод Эйлера

\hs
\begin{gather}
y_{n+1} = y_n + h f(x_{n+1}, y_{n+1})
\end{gather}

для уравнений \hl{1.1} и \hl{1.2}. Уравнение \hl{1.1} линейное, что позволяет найти зависимость

\hs
\begin{gather}
y_{n+1}(z_{n+1}) = \frac{1}{1 - h \alpha(A)} (y_n + h (\beta(A) z_{n+1} + p(x_{n+1}, z_{n+1})))
\end{gather}

и подставить её в нелинейное уравнение \hl{1.2}:

\hs
\begin{gather}
0 = - z_{n+1} + z_n + h \left( \gamma(A) y_{n+1}(z_{n+1}) + \delta(A) z_{n+1} + q(x_{n+1}, y_{n+1}(z_{n+1}), z_{n+1}) \right)
\end{gather}

За одну итерацию необходимо найти корень $ z_{n+1} $ нелинейного уравнения \hl{1.7}, а далее определить $ y_{n+1} $ из \hl{1.6}.

\newpage

\subsection*{Часть 2}
\section*{Реализация}
\resetcounters{2}

\vspace{18pt}

Алгоритм реализован на языке программирования \href{https://julialang.org}{\footnotesize \texttt{Julia}} в виде локального модуля {\footnotesize \texttt{A2}} и расположен в GitHub репозитории \href{https://github.com/paveloom-p/P12}{\footnotesize \texttt{paveloom-p/P12}} в папке \href{https://github.com/paveloom-p/P12/tree/master/A2}{\footnotesize \texttt{A2}}. Для воспроизведения результатов следуй инструкциям в файле {\footnotesize \texttt{README.md}}. Далее приводятся только сниппеты кода.

\subsection{Алгоритм}

\begin{figure}[H]
\begin{lstlisting}[
    caption=Решение системы нелинейных уравнений
]
# Compute the step
h = (b - a) / n

# Prepare the result arrays
y = Vector{Float64}(undef, n+1); z = copy(y)

# Put the boundary values first
y[1] = y₀
z[1] = z₀

# Compute the rest
for i in 1:n
   x = a + i * h
   yᵢ₊₁(x, zᵢ₊₁) = (y[i] + h * (β(A) * zᵢ₊₁ + p(x, zᵢ₊₁))) / (1 - h * α(A))
   z[i+1] = find_zero(
       (zᵢ₊₁) -> -zᵢ₊₁ + z[i] + h *
                 (γ(A) * yᵢ₊₁(x, zᵢ₊₁) + δ(A) * zᵢ₊₁ + q(x, zᵢ₊₁)),
       z[i],
   )
   y[i+1] = yᵢ₊₁(x, z[i+1])
end
\end{lstlisting}
\end{figure}

Функция \href{https://juliahub.com/docs/Roots/o0Xsi/1.0.9/reference/#Roots.find_zero}{\footnotesize \texttt{find\_zero}} взята из пакета \href{https://github.com/JuliaMath/Roots.jl}{\footnotesize \texttt{Roots}}.

\subsection{Пример}

Необходимо решить задачу Коши для жесткой системы уравнений

\hs
\begin{gather}
     y'(x) = \frac{1}{3} y(x) - A z(x) + \sqrt{\frac{z^2}{2}},
\end{gather}

\hsv{10}
\begin{gather}
     z'(x) = -2 y + A z - \frac{z}{z^2 + 1}
\end{gather}

на промежутке $ [0, 1] $ с начальными данными

\hs
\begin{gather}
     y(0) = -4, \; z(0) = 4.
\end{gather}

\begin{figure}[H]
\begin{lstlisting}[
    caption=Определение и решение проблемы
]
y, z = solve(
    Problem(
        (A) -> 1 / 3, # α
        (A) -> -A, # β
        (A) -> -2, # γ
        (A) -> A, # γ
        (x, z) -> sqrt(z^2 / 2), # p
        (x, z) -> - z / (z^2 + 1), # q
        0, # a
        1, # b
        -5, # A
        -4, # y₀
        4, # z₀
    ),
    Options(1000), # n
)
\end{lstlisting}
\end{figure}

\begin{figure}[H]
\captionsetup{justification=centering}
\setlength{\abovecaptionskip}{-10pt}
\renewcommand{\figurename}{Рисунок}
\caption{График решений}
\hspace{30pt}\includegraphics{../figures/result.pdf}
\end{figure}

Смотри численный вывод для данного примера в файле
\href{https://raw.githubusercontent.com/paveloom-p/P12/master/A2/result}{\footnotesize \texttt{result}}.

\end{document}
