\documentclass{article}
\usepackage[english, russian]{babel}
\babelfont{rm}{cmunrm.otf}

\usepackage{amsmath}
\usepackage{relsize}
\usepackage{mathtools}

\usepackage{geometry}
\geometry{a4paper, left=1.2in, right=1in}

\usepackage{titlesec}
\titleformat{\section}
  {\normalfont\fontsize{21}{18}\bfseries}{\thesection}{1em}{}

\usepackage[bookmarks=false]{hyperref}
\hypersetup{pdfstartview={FitH},
            colorlinks=true,
            linkcolor=magenta,
            pdftitle={Вычислительный практикум: отчет по первому заданию},
            pdfauthor={Павел Соболев}}

\usepackage{caption}
\captionsetup[figure]{labelfont=bf,textfont=normalfont,singlelinecheck=off,justification=raggedright}
\setlength{\captionmargin}{1pt}

\usepackage{float}
\input{../../shared/julia_font}
\input{../../shared/julia_listings}
\lstset{language=Julia, style=julia}

\setlength\parindent{0pt}

\newcommand{\hl}[1]{(\hyperlink{eq:#1}{#1})}

\newcommand{\s}[1]{\iter\hypertarget{skip:\thesecounter.\theeqcounter}{\vspace{#1pt}}}
\newcommand{\sd}[0]{\iter\hypertarget{skip:\thesecounter.\theeqcounter}{\vspace{-10pt}}}

\newcommand{\hep}[1]{\vspace{#1pt}\hypertarget{eq:\thesecounter.\theeqcounter}{}\vspace{-#1pt}}

\newcommand{\hs}[0]{\sd\hep{18}}
\newcommand{\hst}[0]{\sd\hep{22}}

\numberwithin{equation}{section}
\newcounter{secounter}[section]
\newcounter{eqcounter}
\newcommand{\iter}[0]{\stepcounter{eqcounter}}
\newcommand{\resetcounters}[1]{\setcounter{section}{#1}
\setcounter{subsection}{0}
\setcounter{equation}{0}
\setcounter{secounter}{#1}
\setcounter{eqcounter}{0}}

\begin{document}

\pagenumbering{Alph}
\thispdfpagelabel{Title}
\begin{titlepage}

    \newcommand{\HRule}{\rule{\linewidth}{0.5mm}}

    \center

    \ \\[6.5cm]

    \textsc{\Large Вычислительный практикум}\\[0.5cm]

    \HRule\\[0.4cm]

    {\huge\bfseries Отчет по первому заданию}\\[0.4cm]

    \HRule\\[0.5cm]

    \large
    \textit{Автор:} Павел \textsc{Соболев}

    \ \\[0.9cm]
    \vfill\vfill\vfill

    {\large\today}

    \vfill

\end{titlepage}
\pagenumbering{arabic}

\subsection*{Часть 1}
\section*{Теория}
\resetcounters{1}

\vspace{18pt}

Необходимо решить линейную краевую задачу

\hs
\begin{gather}
    y''(x) + p(x) y' + q(x) y(x) = f(x)
\end{gather}

с граничными условиями

\hs
\begin{gather}
    y'(a) = \alpha y(a) + A \\
    y'(b) = \beta y(b) + B
\end{gather}

разностным методом с применением метода разностной прогонки.

\subsection{Получение системы линейных уравнений}

Выберем равномерную сетку $ \{x_k\} $:

\hs
\begin{gather}
x_k = a + kh, \; k \in \{0, \dots, n\}, \; h = (b-a)/n.
\end{gather}

Система \hl{1.1} -- \hl{1.3} представима в виде

\hs
\begin{gather}
    \left\{
    \begin{aligned}
        & -b_0 y_0 + c_0 y_1 && = d_0, \\
        a_k y_{k-1} & -b_k y_k +c_k y_{k+1} && = d_k \;\; \forall k \in \{1,...,\,n-1\}, \\
        a_n y_{n-1} & -b_n y_n && = d_n,
    \end{aligned}
    \right.
\end{gather}

где коэффициенты $ a_k, \, b_k, \, c_k, \, d_k $ для $ k\in\{1,...,\,n-1\} $ находятся из разностного соотношения

\hs
\begin{gather}
    a_k = 1 - \frac{h}{2} p_k, \quad b_k = 2 - h^2 q_k, \quad c_k = 1 + \frac{h}{2} p_k, \quad d_k = h^2 f_k,
\end{gather}

а граничные значения $ b_0, \, c_0, \, d_0 $ и $ a_n, \, b_n, \, d_n $ определяются из выбранных разностных соотношений для границ. Используя формулы численного дифференцирования

\hs
\begin{gather}
    y'(x) = \frac{y(x+h)-y(x)}{h} + R, \\
    y'(x) = \frac{y(x)-y(x-h)}{h} + R
\end{gather}

и исключив главные части остатков R, получаем

\hs
\begin{gather}
    y'(a) = \frac{y_1 - y_0}{h} \Big(1 + \frac{h}{2} p_0 \Big) + \frac{h}{2} q_0 y_0 - \frac{h}{2} f_0, \\
    y'(b) = \frac{y_n - y_{n-1}}{h} \Big(1 - \frac{h}{2} p_n \Big) - \frac{h}{2} q_n y_n + \frac{h}{2} f_n.
\end{gather}

Сопоставляя первое равенство из системы \hl{1.5} с \hl{1.9}, определяем, что

\hs
\begin{gather}
    -b_0 = -\frac{1}{h} \left( 1 + \frac{h}{2} p_0 \right) + \frac{h}{2} q_0 - \alpha, \\
    c_0 = \frac{1}{h} \left( 1 + \frac{h}{2} p_0 \right), \\
    d_0 = \frac{h}{2} f_0 + A.
\end{gather}

Аналогично, сопоставляя последнее равенство из системы \hl{1.5} с \hl{1.10}, получаем, что

\hs
\begin{gather}
    a_n = \frac{1}{h} \left( \frac{h}{2} p_n - 1 \right), \\
    -b_n = -\frac{1}{h} \left( \frac{h}{2} p_n - 1 \right) - \frac{h}{2} q_n - \beta, \\
    d_n = B - \frac{h}{2} f_n
\end{gather}

Таким образом, все коэффициенты системы \hl{1.5} определены.

\subsection{Решение системы линейных уравнений}

Метод прогонки осуществляется в два этапа. \par

\vspace{\baselineskip}

Прямой ход: для $ i \in \{2, \dots, n\} $

\hs
\begin{gather}
    w_i = \frac{a_i}{b_{i-1}}, \\
    b_i = b_i - w_i c_{i-1}, \\
    d_i = d_i - w_i d_{i-1},
\end{gather}

и обратный ход:

\hs
\begin{gather}
    y_n = \frac{d_n}{b_n}, \\
    y_i = \frac{d_i - c_i y_{i+1}}{b_i} \;\; \forall i \in \{n-1, n-2, \dots, 1\}.
\end{gather}

\newpage

\subsection*{Часть 2}
\section*{Реализация}
\resetcounters{2}

\vspace{18pt}

Алгоритм реализован на языке программирования
\href{https://julialang.org}{Julia} и расположен в GitHub репозитории \href{https://github.com/paveloom-p/P12}{paveloom-p/P12} в папке \href{https://github.com/paveloom-p/P12/tree/master/A1}{A1}. Далее приводятся только сниппеты кода.

\subsection{Получение системы линейных уравнений}


\begin{figure}[H]
\begin{lstlisting}[
    caption=d
]
# Вычисли шага
h = (b - a) / n

# Вычисли вспомогательные переменные
h⁻¹ = 1 / h
hₕ = h / 2
ax₀ = h⁻¹ * (hₕ * p(a) + 1)
axₙ = h⁻¹ * (hₕ * p(b) - 1)

# Вычисли диагонали и правый вектор-столбец
dl = OffsetArray([[1 - hₕ * p(a + k * h) for k in 1:n-1]; axₙ], 2:n+1)
d = [
        -ax₀ + hₕ * q(a) - α;
        [h^2 * q(a + k * h) - 2 for k in 1:n-1];
        -axₙ - hₕ * q(b) - β
    ]
du = [ax₀; [1 + hₕ * p(a + k * h) for k in 1:n-1]]
f = [hₕ * f(a) + A; [h^2 * f(a + k * h) for k in 1:n-1]; B - hₕ * f(b)]
\end{lstlisting}
\end{figure}

\end{document}
